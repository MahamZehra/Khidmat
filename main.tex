\documentclass{article}

\usepackage{geometry}
\usepackage{hyperref}

\title {Khidmat: Manual Coding}

\author{
  Maham Zehra\\ mz04027
}
\date{}  

\begin{document}
\maketitle

% Use first person plural (we, us) even if you did the Khidmat individually.

% An introduction of the project, no more than 2 sentences. Provide the highest level of detail only. Other details will come later.
% Typically, "This project is to <short description of project> for/at <client>."
%This project is to build a testing system to be used for the entrance examination at Habib University.
\noindent
This project aims to teach children the basics and concepts of coding, using hands-on activities.
\vspace{2.5mm}

% About the client.
%Habib University is a first of its kind liberal arts and science university in Pakistan. It offers 4 major and several minor programs. Founded through the largest philanthropic grant to higher education in the history of Pakistan, it is a registered non-profit organization with \href{http://www.pcp.org.pk}{Pakistan Centre for Philanthropy (PCP)}.
\noindent
 Hussaini Foundation is a registered NGO with the Government of Pakistan. The Child Development Project (CDP) under the Foundation is primarily involved in conducting certified training programmes for parents, teachers and youths.\cite{cdp-hf} 
 \vspace{2.5mm}

% About the project.
%Striving to admit the most deserving students, Habib University uses \href{https://accuplacer.collegeboard.org}{ACCUPLACER} for its entrance test but wants to move to its own test. Faculty members will contribute questions to a pool and the new examination system will choose and present questions at random from the pool to each test taker. This will ensure that each test taker gets a unique test. For our Khidmat, we will build Habib University's new examination system.
\noindent
Striving to provide learning and learned individuals a platform to either gain more or spread their knowledge, we were given such an opportunity to transfer our knowledge and understanding of programming to children of the age group 8-12. 
\vspace{2.5mm}

% About the plan of work.
%We will work full time on Habib University's premises under the supervision of their Admission Manager. The goal is to develop, test, and deploy the system by the end of our Khidmat.
\noindent 
We conducted 1.5 hour class everyday from Tuesday, 17th, to Friday, 20th, where we did multiple hands on activities like Lego programming, a game of If-Then, sending encrypted messages to each other, creating a keepsake of their names using binary ASCII codes. \\ We then conducted a 1.5 hour class on Saturday, 21st and a 2.5 hour class on Monday, 23rd, where the children implemented all the concepts and logic they had built on an online python compiler.
The goal of the course is that the children have enough know-how that they can further increase their command on python, following the tutorials that have been shared with them.

% Copy-paste this section with necessary modifcations for each week.
\newpage % Start the report for each week on a new page.
\section*{Week 1: 17--20 July, 2018}

% A summary, maximum 2 sentences, of this week's activities.
%We spent this week meeting several stakeholders in order to understand the shortcomings of ACCUPLACER and the expectations from the new system.
We spent this week trying to introduce elements of programming, i.e. variables, constants, keywords, functions, loops, conditionals.
\vspace{2.5mm} \\
\begin{tabular}{|l|l|l|l|}
  \hline
  Item 	& Activity & Time & ID \\\hline\hline
  1	& Taught the children & 6 hrs & mz04027 \\\hline
  %2	& Met faculty & 2 hrs & st2 \\\hline
  %3	& Met IT team & 3 hrs & st1, st2 \\\hline
  %$\vdots$ & $\vdots$ & $\vdots$ & $\vdots$\\\hline
\end{tabular}
\vspace{5mm} \\
The total time spent on the Khidmat this week is as follows.
\vspace{2.5mm} \\
\begin{tabular}{|l|l|}
  \hline
  ID & Total Hours\\\hline\hline
  mz04027 & 6 hours\\\hline
  %st2 & 6 hours\\\hline
\end{tabular}

% Other weeks ...
%\newpage
\section*{Week 2: 21--23 July, 2018}
We spent this week trying to teach the students the syntax of the concepts they had learned earlier.
\vspace{0.01mm} \\
\begin{tabular}{|l|l|l|l|}
  \hline
  Item 	& Activity & Time & ID \\\hline\hline
  1	& Taught the children & 4 hrs & mz04027 \\\hline
\end{tabular}
\vspace{5mm} \\
The total time spent on the Khidmat this week is as follows.
\vspace{2.5mm} \\
\begin{tabular}{|l|l|}
  \hline
  ID & Total Hours\\\hline\hline
  mz04027 & 4 hours\\\hline
\end{tabular}
\newpage
\section*{Conclusion}

% Remind the reader about the project. Summarize your activities over the course of the project.
%Our project was to build a new testing system for Habib University to replace ACCUPLACER for its entrance examination. We started by meeting all the stakeholders to understand their expectations from the new system. We then identified the necessary tools to build the required system and trained ourselves on them. Development and testing were carried out in collaboration with the IT team so that any shortcomings were identified and catered to as we went along. The system was then deployed and officers from the Admissions Team were trained to use it.
Our project was to conduct a workshop/course on coding, using hands-on activities. \\
We started by explaining the concept of number systems, binary and decimal, their conversions, ASCII codes, and then the children made a keepsake of their names using binary ASCII codes of each letter; the children also conversed in encrypted codes, instead of writing the letter they wrote its ASCII code, and then the partner would decry-pt the message and answer in encrypted form again. \\
We then explained the concept of variables, constants, keywords, loop, conditionals and functions; and then played a game of If-Then with increasing levels, made a maze and coded our way out of it. \\
We also used an online python compiler to familiarize the children with syntax and different types of error. \\
We shared an online tutorial websites and compiler with them, that they can install to further improve their understanding, and for any queries they can also contact us.

\newpage
\thispagestyle{empty}
% Show your external supervisor your report, especially the weekly updates; have them sign a printed copy of this page; scan the signed page; and include the scanned page in this document as an image.

\begin{center}
  {\Large\bf Khidmat Completion Form}\\[5pt]
  \small To be completed by the external supervisor.  
\end{center}
\bigskip

\noindent{\it Please use the space below to provide any comments you may have on the students' performance, the Khidmat program, or any other feedback you want to share with Habib University's Khidmat committee. We can also be reached at \href{mailto:khidmat@sse.habib.edu.pk}{khidmat@sse.habib.edu.pk}.}
\vfill

\begin{center}
  \rule{.8\textwidth}{.5pt}
\end{center}
\medskip

% Insert your name below.

I hereby certify that I supervised Maham Zehra for the Khidmat described in this report. Furthermore, that I have read and agree with the weekly updates included in this report. My signature below marks the successful completion of the work undertaken for the Khidmat.\\
\bigskip
\bigskip

\noindent\begin{tabular}{@{}p{.6\textwidth}@{\hspace{.1\textwidth}}p{.3\textwidth}}
  \hrulefill &   \hrulefill \\
  Name and signature & Location and date
\end{tabular}
\begin{thebibliography}{2}
\bibitem{cdp-hf}
About Child Development Programme \\ http://cdp-hf.org/child-development-program/
\end{thebibliography}
\end{document}
